
\renewcommand{\abstractname}{Abstract}
\begin{abstract}
\noindent The rate of catalytic material undergoes reduction activities by various
mechanisms, one being the poisoning of catalytic regions by adsorbed chemicals,
reagents pass so that it does not reach more such regions. This work is a
simulation whose main objective It is to replicate the results already achieved
in previous jobs, in addition to results coming simulation also show approximate
analytical results. O simulated model is a system composed by a one-dimensional
network with catalytic sites and reagents evenly distributed, where such a
network is the stage unimolecular reactions. The dynamics of the model is
through the dissemination of reagents possibly reach a catalytic site, react
instantly and leave the system in the form of product, and this reaction can
disable catalytic site with probability \ textit {p}. The system has 2 phases
different: one in which a finite positive number of reagents survives dynamics;
another in which no deactivation of all the catalytic sites of the network. The
system dynamics is governed by bimolecular reaction where catalytic sites play a
role in the second reactive species, when in the border region between these
phases. The behavior near criticality system has intersection where he obeys a
power law with exponent close $-1/4$ To time in the order of $10^3$.
\hfill \break
\hfill \break
\textbf{Keywords:} catalytic deactivation, discrete networks, simulation.
\end{abstract}
