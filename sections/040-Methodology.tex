
\section{Metodologia}

Para a execução da simulação, um primeiro passo é definir aspectos do ambiente
simulacional.

A implementação da rede unidimensional do se sistema se dá por um arranjo, onde
cada posição do arranjo pode: conter um sítio inativo desocupado; conter um
sítio inativo ocupado por um reagente; conter um sítio ativo.

Três listas encadeadas são utilizadas como estruturas auxiliares para
distribuir uniformemente os sítios ativos e os reagentes na rede. Um sítio
aleatoriamente escolhido é removido de uma primeira lista auxiliar, de forma a
garantir que um sítio não seja escolhido duas vezes, enquanto uma segunda lista
auxiliar mantém referência para todos os sítios da rede. Uma terceira lista
armazena a sequência de posições dos sítios aleatóriamente escolhidos para
reagentes, com finalidade de facilitar a simulação do processo de difusão. A
lista encadeada auxiliar que mantém referência de todos os sítios da rede é
mapeada em um arranjo linear após distribuição aleatória de sítios catalíticos e
reagentes.

São parâmetros de simulação: número de execuções independentes de cada dinâmica
$M$; duração da dinâmica $Z$, em termos do número de tentativas de se executar
um passo, onde cada reagente do sistema tenta executar $Z$ passos; o tamanho da
rede $N$, correspondente ao número de sítios contidos por ela; e coeficiente de
difusão $\mathcal{D}$, correspondente ao número de tentativas de se executar um
passo do tamanho da unidade da rede, por unidade de tempo, por reagente.

O gerador de números pseudo-aleatórios utilizado nas simulações é o algoritmo
\texttt{Ranq1} \cite{8}. Cada número pseudo-aleatório gerado depende do número
gerado anteriormente, sendo que o primeiro elemento depende do valor fornecido
como semente ao algoritmo. Essa característica é determinante para que todas as
$M$ execuções independentes aconteçam sequencialmente, sem que um i-ésimo valor
pseudo-aleatório gerado seja considerado mais de uma vez ou desconsiderado.

A difusão de reagentes respeita o princípio de volume excluído, de forma que se
determinado reagente tenta executar passo com destino a um sítio previamente
ocupado, tal tentativa é rejeitada e o reagente não se desloca nessa iteração.
Os resultados da simulação são: o número de passos executados por cada reagente
e a dupla composta pela contagem de reagentes e pela contagem de sítios
catalíticos, por iteração da rede e por execução independente, correspondentes a
um total de $M \times Z$ duplas.

Os valores dos parâmetros utilizados na simulação foram: $N = 2^{20}$,
$\mathcal{D} = 1$, $Z = 10^8$.
