
\renewcommand{\abstractname}{Resumo}
\begin{abstract}
\noindent Este trabalho trata de uma simulação cujo principal objetivo é
replicar resultados já alcançados em trabalhos prévios. O modelo simulado trata
de um sistema composto por uma rede unidimensional, com sítios catalíticos e
reagentes uniformemente distribuídos. A dinâmica do modelo se dá através da
difusão dos reagentes que possivelmente atingem um sítio catalítico, reagem
instantaneamente e deixam o sistema na forma de produto, sendo que essa reação
pode desativar o sítio catalítico com probabilidade \textit{p}. O sistema
apresenta uma fase em que um número positivo finito de reagentes sobrevive à
dinâmica; outra em que há desativação de todos os sítios catalíticos da rede. A
dinâmica do sistema é regida por reação bimolecular, onde sítios catalíticos
desempenham papel da segunda espécie reativa, quando na região de fronteira
entre tais fases. O comportamento do modelo próximo da criticalidade apresenta
cruzamento, onde ele obedece lei de potência com expoente próximo de $-1/4$ até
tempos da ordem de $10^3$.
\hfill \break
\hfill \break
\textbf{Palavras-chave:} desativação catalítica, redes discretas, simulação.
\end{abstract}
