
\section{Introdução}

\lettrine{N}{este} trabalho, é realizada uma simulação de um processo de
reação-difusão com atividade catalítica a partir de modelos desenvolvidos
em \cite{3}.

Esses modelos de reação-difusão são sistemas que envolvem reagentes sendo
convertidos em produtos por uma reação química e transportados no espaço pela
difusão. Esse tipo de sistema é muito estudado em engenharia química, mas ocorre
frequentemente também em outras áreas \cite{4}.

A simulação do trabalho possui como objetivo, a reprodução de resultados já
alcançados por meio desses modelos e propiciar o aprendizado e desenvolvimento
desse método numérico de modelagem de sistemas reais.

Simulação é uma forma de modelagem utilizada na análise de sistemas complexos
que envolve, portanto, o desenvolvimento de um modelo matemático que, após a
prática de experimentações deste modelo , possibilita a obtenção de um histórico
de comportamento do sistema ao longo do tempo e de estatísticas desse
comportamento \cite{1}.

O modelo matemático, por sua vez, pode ser definido como ``uma
representação, em termos matemáticos, do comportamento de objetos e dispositivos
reais'' \cite{2}. Esse modelo pode ser analítico ou de
simulação, que diferem pela natureza de suas soluções. Modelos analíticos
possuem como solução, uma equação fechada que descreve o comportamento do
sistema ao longo do tempo \cite{1}.

Modelos de simulação são analisados por meio da execução de um programa que
gera amostras do comportamento do sistema, que permite realização de análises
do sistema \cite{1}.

O problema a ser tratado pela simulação é a análise comportamental da reação de
armadilhamento com desativação ao longo do tempo. Para uma formulação apropriada
deste problema e o posterior desenvolvimento de um modelo conceitual para tal
sistema, primeiro deve ser analisado o problema de forma mais simples, sem
desativação da armadilha, cujo modelo já possui solução analítica na
bibliografia. Esse modelo corresponde a uma versão discreta do modelo de
Smoluchowski \cite{5}.
