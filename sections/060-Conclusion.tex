
\section{Conclusão}

As simulações realizadas descreveram bem a dinâmica dos reagentes em diferentes
regimes. Foi possível perceber que, como esperado, quando
$\sigma_0 > p \theta_0$, os reagentes tendem a diminuir de forma exponencial até
que todos abandonem o sistema, virando produtos. Outro regime ocorre quando
$\sigma_0 < p \theta_0$ e, então, como foi possível perceber, os reagentes
apresentam, inicialmente, uma queda muito acelerada, mas os catalisadores também
são envenenados em uma frequência alta até que todos sejam desativados e, como
resultado, alguns reagentes permanecem no sistema por tempo indeterminado, sem a
possibilidade de reagirem. Intermediando os dois regimes, existe um ponto
crítico de transição de fases em que os números de reagentes e catalisadores
decaem proporcionalmente no sistema.

A simulação propiciou, portanto, uma análise qualitativa da dinâmica do sistema.
Dessa forma, é possível perceber a importância dessa forma de modelagem,
principalmente em sistemas em que não se consegue por alguma forma, obter alguma
solução analítica. Além disso, a concordância dos resultados simulacionais com
equações modeladas para casos específicos do sistema, reforçam a validade do
modelo proposto.
