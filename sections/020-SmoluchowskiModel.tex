
\section{Modelo de Smoluchowski}

Smoluchowski propôs uma modelagem do processo de armadilhamento a partir de uma
armadilha ou catalisador esférico de raio $r$, no qual as partículas reagem e saem
do sistema instantaneamente ao tocar a borda dessa esfera. Inicialmente, as
partículas estão distribuídas com concentração uniforme $\rho_0$ de partículas
ao redor da esfera \cite{3}. A variação da concentração dos reagentes em um
ponto $x$ fora da esfera ao longo do tempo $t$, $\rho(x,t)$, é dada pela
equação \cite{3}:

{
\setlength{\belowdisplayskip}{0pt} \setlength{\belowdisplayshortskip}{0pt}
\setlength{\abovedisplayskip}{0pt} \setlength{\abovedisplayshortskip}{0pt}

\begin{equation}
  \dfrac{\partial}{\partial t}\rho(x,t) = \mathcal{D}\nabla^2\rho(x,t), x>r,
  \label{Equation-021}
\end{equation}
}

\noindent sendo $\mathcal{D}$ o coeficiente de difusão dos reagentes. E já que os
reagentes que tocam a borda da esfera saem do sistema instantaneamente, a
densidade de partículas nessa região é 0, ou seja \cite{3}:

{
\setlength{\belowdisplayskip}{0pt} \setlength{\belowdisplayshortskip}{0pt}
\setlength{\abovedisplayskip}{0pt} \setlength{\abovedisplayshortskip}{0pt}

\begin{equation}
  \rho(x,t)|_{x=r} = 0
  \label{Equation-022}
\end{equation}
}

Tornando o problema discreto, pode-se considerar o sistema como uma rede de
sítios em que cada um desses sítios pode estar vazio, ocupado por um reagente
ou por um catalisador \cite{3}.  Dessa forma, definindo $c$ como a concentração de
armadilhas na rede e $S(t)$ o número de sítios distintos visitados por um reagente
até o tempo $t$, a probabilidade de sobrevivência $P_s$ de sobrevivência de uma
partícula no instante $t$ é \cite{6}:

{
\setlength{\belowdisplayskip}{0pt} \setlength{\belowdisplayshortskip}{0pt}
\setlength{\abovedisplayskip}{0pt} \setlength{\abovedisplayshortskip}{0pt}

\begin{equation}
  P_s(t) = \left<(1-c)^{S(t)}\right>
  \label{Equation-023-Probabilidade}
\end{equation}
}

A equação \ref{Equation-023-Probabilidade} é difícil de ser solucionada
analiticamente \cite{3}. Mas caso fosse solucionada, a densidade dos reagentes no
tempo $t$ poderia ser obtida da seguinte forma:

{
\setlength{\belowdisplayskip}{0pt} \setlength{\belowdisplayshortskip}{0pt}
\setlength{\abovedisplayskip}{0pt} \setlength{\abovedisplayshortskip}{0pt}

\begin{equation}
  \rho(t) = \rho_0P_s(t)
  \label{Equation-024}
\end{equation}
}

Para tempos curtos, pode-se utilizar a seguinte aproximação para a equação
\ref{Equation-023-Probabilidade}:

{
\setlength{\belowdisplayskip}{0pt} \setlength{\belowdisplayshortskip}{0pt}
\setlength{\abovedisplayskip}{0pt} \setlength{\abovedisplayshortskip}{0pt}

\begin{equation}
  P_s(t) \approx (1-c)^{\left<S(t)\right>}
  \label{Equation-025}
\end{equation}
}

Para definir essa concentração em tempos longos, é feito uma outra simplificação
no modelo, tornando-o unidimensional. Dessa forma, existe um espaço $L$ entre duas
armadilhas consecutivas no qual os reagentes podem circular livremente sem que
um sítio ocupe mais de um reagente. Para cada um desses espaços, sabe-se que a
concentração de reagentes nos sítios ocupados por armadilhas (0 e $L$) são sempre
0. Em um sítio intermediário $x$, a concentração inicial de reagente é $\rho_0$ e,
ao longo do tempo, essa concentração diminui com a seguinte taxa \cite{3}:

{
\setlength{\belowdisplayskip}{0pt} \setlength{\belowdisplayshortskip}{0pt}
\setlength{\abovedisplayskip}{0pt} \setlength{\abovedisplayshortskip}{0pt}

\begin{equation}
  \dfrac{\partial \rho}{\partial t} = \mathcal{D}\dfrac{\partial^2 \rho}{\partial x^2}
  \label{Equation-026-TempoLongo}
\end{equation}
}

A equação \ref{Equation-026-TempoLongo} possui uma solução analítica aproximada,
como segue \cite{3}:

{
\setlength{\belowdisplayskip}{0pt} \setlength{\belowdisplayshortskip}{0pt}
\setlength{\abovedisplayskip}{0pt} \setlength{\abovedisplayshortskip}{0pt}

\begin{equation}
  \rho(t) \approx exp\left[ -\dfrac{3}{2} (2\pi^2c^2\mathcal{D}t)^{1/3} \right]
  \label{Equation-027-Decaimento}
\end{equation}
}
